\documentclass{article}

\usepackage{amsmath}
\usepackage{algorithm}
\usepackage{algpseudocode}
\usepackage[thmmarks,  thref, amsmath]{ntheorem}

\theoremstyle{plain}
        \newtheorem{theorem}{Theorem}

        \theoremheaderfont{\itshape}
        \theorembodyfont{\upshape}
        \newtheorem{case}{Case}

        \theoremstyle{nonumberplain}
        \theoremheaderfont{\scshape}
        \theorembodyfont{\upshape}
        \theoremsymbol{\scshape Q. E. D.}%\ensuremath{ _{\box}}
        \theorempostwork{\setcounter{case}{0}}
        \newtheorem{proof}{Proof}

\title{Applied Linear Algebra Homework One}
\date{2017-04-07}
\author{Jesse Bannon}

\begin{document}
\maketitle
  \section{Matrices Excersize}
  
  \begin{theorem}
  Let $A$ be an adjacency matrix for a graph $G$. $A_{i,j}^p$ represents the number of walks of length $p$ from vertex $i$ to $j$, $i, j \in G$.
  \end{theorem}
  
  \begin{proof}
        We prove that an adjecency matrix $A$ for a graph $G$ is nilpotent if and only if $G$ is a directed acylic graph using a case proof that considers all possible graphs.
        \begin{case}
        	$A = 0$ \\
        	
        	We can consider $G$ to be a DAG since $G$ has no cycles, and $A^p = 0$ for $p > 0$.
        	
        	Thus $A$ is nilpotent and $G$ is a DAG.
        \end{case}
        
        \begin{case}
        	$G$ contains a cycle \\
        	
        	Using \textbf{Theorom 1}, we know that $A^p \neq 0$ for $p > 0$. \\

        	Let $C$ be the transitive closure from vertex $i$ to $i$, where $i \in G$ has a cycle.
        	
			There exists a walk of $p$ steps from vertex $i$ to some vertex $j \in C$ for $p > 0$. \\

        	Thus, $\exists A_{i,j}^p \in A^p \to A_{i,j}^p > 0$ for $p > 0$. 
        	
        	Thus $A$ is not nilpotent and $G$ is not a DAG.
        \end{case}
        
        \begin{case}
        $G$ contains no cycles \\
        
        Using \textbf{Theorom 1}, we know that $A^p = 0$ for some $p > 0$.
        
        Let $n = |V(G)|$ \\
        
        Since there are no cycles, $G$ is considered a DAG. The largest path from 
        
        any vertex $i \in G$ can only contain every other vertex $\in G$. Thus, the largest 
        
        possible number of walks for any path $\in G$ can only be $n-1$. \\
        
        Thus, $A^n = 0$, making $A$ nilpotent. $A$ is always nilpotent if $G$ is a DAG.
        \end{case}
   \end{proof}
    Thus, $A$ is nilpotent if and only if $G$ is a DAG. We can also conclude that the smallest value of $p$ for which $A^p = 0$ indicates that the largest path $\in G$ contains $p - 1$ vertices.
\end{document}